%-------------------------------------------------------------------------------
%	SECTION TITLE
%-------------------------------------------------------------------------------
\cvsection{Professional Experiences}


%-------------------------------------------------------------------------------
%	CONTENT
%-------------------------------------------------------------------------------
\begin{cventries}

%---------------------------------------------------------

\cventry
{Principal Data Scientist \textit{ -- Leverage advanced analytics to optimize supply chain of Walmart US}}          % Job title
{Walmart Inc.}                      % Organization
{Bentonville, AR}                   % Location
{Apr 2018 - Present}                    % Date(s)
{
    \begin{cvitems}
    \item {
        \textbf{Delivery Window Optimization}  
        % {\small{}}   
        \begin{itemize}
            \item  Reduced the length of time window of store delivery from >12hrs to around 4hrs by optimization considering
             % (\textit{C++, CPLEX, Python}) 
             multiple constraints, such as tranportation cost minimizing, DC workload balancing, noise ordinance obedience, etc. 
        \end{itemize}           
    }
    \item {
        \textbf{Store-to-DC Alignment Optimization} 
        % {\small{}}
        \begin{itemize}
            \item  Created a web-app to support DC-to-store alignment, with goals of minimizing transportation cost and avoiding unnecessary / undesired impacts. 
            (Model: \textit{C++, CPLEX, Python}, UI: \textit{Python [Django], JavaScript [RequireJS, D3.js]}) 
        \end{itemize}           
    }
    \item {
        \textbf{Return Decision Module (RDM)} 
        % {\small{[]}}
        \begin{itemize}
            \item  In charge of middle-ware of RDM -- a decision support tool that helps handle the merchandise return in the most economical way. 
                   (\textit{Java [Tomcat, Spring, Camel, ActiveMQ, Jersey, Guice, Dozer], SQL [DB2, SQL Server, TeraData]})
        \end{itemize}           
    }
    \end{cvitems}% Description(s) of tasks/responsibilities 
}

%---------------------------------------------------------

\cventry
{Sr. Manager of Operations Research \textit{ -- Dedicated to creating decision support tools for railroads}}            % Job title
{CSX Transportation Inc. }                % Organization
{Jacksonville, FL}                       % Location
{Mar 2012 - Apr 2018}                     % Date(s)
{
    \begin{cvitems}
    \item {
        \textbf{Train Planner} 
        % {\small{[Python (Django, Scikit-learn), JavaScript]}}
        \begin{itemize}
            \item  Combined optimization, data-mining, and simulation to create a web-based decision support tool to schedule / simulate / visualize the meet-and-pass (time \& location) of line-of-road trains 
            \item  Used widely by multiple departments of CSX -- Service Planning: test and evaluate new plans; Engineering: find proper time windows for maintenance; Network operation: perform capacity and impact analysis, etc.
        \end{itemize}           
    }
    \item {
        \textbf{Line-of-road Emulator} 
        %{\small{[Java (Tomcat), JavaScript (jQuery, D3) / AnyLogic \texttrademark]}}
        \begin{itemize}
        \item  Created a web-based tool to identify network bottlenecks and to evaluate impacts of trains by dynamic visualization. 
               Developed data preloading and caching modules to improve the running efficiency and user experiences
        \item  Provided two levels of view to monitor train movements on a GIS map  
        \begin{itemize}
            \item  \textit{Macro view overlooks the whole network}.  
                   It animates trains as moving dots, finds and marks out problematic ones (e.g., slow). 
                   It was used to illustrate to executive team and board members how congestion happened on the northern tier corridor in 2014 winter, 
                   which received high appreciation and won CSX Spotlight award
            \item  \textit{Micro view focuses on local}. It shows trains as moving strips, which enables users observing more details en route. 
                   It was used to analyze the blockage to road traffic in Elsdon, IL to dispute the accusation from TRB
            \end{itemize}
        \end{itemize} 
	}
    \item {
        \textbf{Hump Yard Simulation System (HYSS)} 
        % {\small{[Microsoft C\#, Access, SQL Server, R]}}
        \begin{itemize}
        \item  Project manager. Developed data feeding module. Designed and performed statistical tests to validate output of HYSS,
              which consists of 6 hump yard models simulating their processes on receiving, classifying and departing
              % (receiving, classifying and departing), and to predict yard status under given inputs
              % a tool to speed up data preparation (from 6 weeks to 3 days) and 
        \item  Assisted network modeling team using HYSS to perform various what-if analysis and to make strategic and tactical decisions (e.g., planning track utilization and right-sizing of yard resources)
        \end{itemize}     
    }
    \end{cvitems}
}

%---------------------------------------------------------

%\cventry
%{Research/Teaching Assistant}          % Job title
%{University of Pittsburgh}             % Organization
%{Pittsburgh, PA}                       % Location
%{Sep 2005 - Feb 2012}                  % Date(s)
%{
%    \begin{cvitems}                          
%        \item {
%Developed methodology to efficiently identify the best response strategies for PEMA (Pennsylvania %Emergency Management Agency, funding provider) by integrating agent-based simulation (\textit{Java %+ RePast}) with advanced statistical selection algorithm (\textit{R}).
%		}
%        \item {
%Taught/assisted undergraduate courses: IE-1083: Simulation with Arena (Arena\markright, 40+ %students, 2011), IE-1051: Computer Aided Design (SolidWorks\markright, 30+ students, 2005).
%		}        
%    \end{cvitems}% Description(s) of tasks/responsibilities  
%}


%---------------------------------------------------------
\cventry
{Software Engineer \textit{ -- Designed and Developed J2EE-based web-apps}}                   % Job title
{COFCO Group}                         % Organization
{Beijing, China}                      % Location
{Jul 2003 -- Jul 2005}                % Date(s)
{
    \begin{cvitems}                          
    \item {
        \textbf{Agricultural Information Subscribing Platform} 
        \begin{itemize}
            \item  Developed an information providing system that allows users subscribing agricultural information via cellphone short-messages
                   (collaborative project between COFCO, CCTV-7 and China National Radio)
                   % (\textit{J2EE/JSP/Servlet, Struts, Hibernate, XML, SQL})
        \end{itemize} 
	}          
    \item {
        \textbf{Employee Satisfaction Survey System }
        \begin{itemize}
            \item  Designed and developed a web-based survey system for COFCO HR. 
                   Used in 2005 and 2006 at COFCO headquarter and its two subsidiaries ($\sim$500 employees involved)
        \end{itemize} 
	}        
    \end{cvitems}  % Description(s) of tasks/responsibilities  
}
%---------------------------------------------------------
\end{cventries}
